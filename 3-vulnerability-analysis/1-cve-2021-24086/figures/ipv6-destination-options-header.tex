\documentclass{article}
\usepackage{import}
\subimport{../../../}{preamble.tex}
\standalonetrue
\begin{document}
\begin{figure}[H]
	\begin{center}
		\begin{bytefield}[bitwidth=1em]{32}
			\bitheader{0-31}\\
			\begin{rightwordgroup}{Destination Options header}
                \bitbox{8}{Next header} & \bitbox{8}{Hdr Ext Len} & \bitbox[tlr]{16}{} \\
                \wordbox[lr]{2}{Options} \\
				\wordbox[lrb]{1}{$\cdots$}
			\end{rightwordgroup}
			\\
			\\
			\begin{rightwordgroup}{Options}
                \bitbox{8}{Option type} & \bitbox{8}{Opt Data Len} & \bitbox[tlr]{16}{} \\
				\wordbox[lr]{1}{Option Data} \\
				\wordbox[lrb]{1}{$\cdots$}
			\end{rightwordgroup}
		\end{bytefield}
	\end{center}

	Where

	\begin{description}
        \item[Next Header] is an 8-bit selector identifying the initial header type of the Fragmentable part of the original packet.
        \item[Hdr Ext Len] is an 8-bit unsigned integer describing the length of the Destination Option header in 8-octets units excluding the first 8 octets
        \item[Options] is a variable-length field. See below
	\end{description}

	And

	\begin{description}
        \item[Option Type] is an 8-bit identifier of the option type
        \item[Opt Data Len] is an 8-bit unsigned integer describing the length of the \emph{Data Option} field in octets
        \item[Options] is a variable-length field with data specified by the option type 
	\end{description}

	\caption{IPv6 Destination Options Header \cite[sec. 4.6]{url:rfc:ipv6}}
	\label{fig:ipv6-destination-options-header}
\end{figure}
\end{document}