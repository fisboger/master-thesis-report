\documentclass{article}
\usepackage{import}
\subimport{../../../}{preamble.tex}
\standalonetrue
\begin{document}
\begin{figure}[H]
	\begin{center}
		\begin{bytefield}[bitwidth=1em]{32}
			\bitheader{0-31}\\
			\begin{rightwordgroup}{Fragment header}
                \bitbox{8}{Next header} & \bitbox{8}{Reserved} & \bitbox{13}{Fragment offset} & \bitbox{2}{Res} & \bitbox{1}{M} \\
                \bitbox{32}{Identification}
			\end{rightwordgroup}
		\end{bytefield}
	\end{center}

	Where

	\begin{description}
        \item[Next Header] is an 8-bit selector identifying the initial header type of the Fragmentable part of the original packet.
        \item[Reserved] is an 8-bit reserved field. Initialized to zero.
		\item[Fragment Offset] is a 13-bit unsigned integer stating the offset.
		\item[Res] is a 2-bit reserved field that is initialized to zero by the transmitter and ignored by the receiver.
		\item[M flag] is a 1-bit boolean field describing if this is the last fragment. 1 = more fragments, 0 = last fragment.
		\item[Identificiation] is a 32-bit identifier that is unique to fragments from the same package.
	\end{description}
	% \caption{\gls{nbns} Name Query Request fields}
	% \label{list:nbns-name-query-request-fields}


	\caption{IPv6 Fragment Header \cite[sec. 4.5]{url:rfc:ipv6}}
	\label{fig:ipv6-fragment-headert}
\end{figure}
\end{document}