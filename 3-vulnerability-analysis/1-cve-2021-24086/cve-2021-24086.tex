\documentclass{report}
\usepackage{import}
\subimport{../../}{preamble.tex}
\standalonetrue
\begin{document}

% sample content
\section{CVE-2020-24086}
\label{sec:CVE-2020-24086}
According to Microsoft\cite{url:microsoft:cve-2021-24086} CVE-2021-24086 is a denial of service vulnerability with a CVSS:3.0 score of 7.5 / 6.5, that is a base score metrics of 7.5 and a temporal score metrics of 6.5. The vulnerability affects all supported versions of Windows and Windows Server. According to an accompanied blog post published by Microsoft
\cite{url:microsoft:cve-2021-24086-blog} at the same time as the patch was released, details that the vulnerable component is the Windows TCP/IP implementation, and that the vulnerability revolves around IPv6 fragmentation. The Security Update guide and the blog post also \todo{Figure out if this should be here}present a workaround that can be used to temporarily mitigate the vulnerability by disabling IPv6 fragmentation.

\subsection{Public information}
Due to the \gls{mapp}\cite{url:microsoft:mapp} security software providers are given early access to vulnerability information. This information often include \gls{poc}s for vulnerabilities to be patched, in order to aid security software providers to create valid detections for exploitation of soon-to-be patched vulnerabilities. Due to \gls{mapp}, some security software providers publish relevant information regarding recently patched vulnerabilities. However, the information is usually very vague in details, and can therefore only aid in the initial exploration of the vulnerability. For CVE-2021-24086, both McAfee\cite{url:mcafee:cve-2021-24086} and Palo Alto\cite{url:palo-alto:cve-2021-24086}

\subsection{Patch diffing}
\subsection{Root-cause analysis}
\subsection{Triggering the vulnerability}

% \subimport{}{subsubsection.tex}

\end{document}