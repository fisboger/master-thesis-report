\documentclass{report}
\usepackage{import}
\subimport{../}{preamble.tex}
\standalonetrue
\begin{document}


\section{Related work}
Using host-based telemetry gathering methods to detect exploitation of vulnerabilities, both known and unknown, has been widely discussed before\cite{paper:related-work:taint-check}\cite{paper:related-work:dacoda}. One example hereof is DACODA\cite{paper:related-work:dacoda}, which traces a network packet through the relevant processes until an unintended memory state happens, such as jumping to an address present in the packet.
\\
Many \gls{NDR} products claim to be able to detect exploitation of both known vulnerabilities and zero-days\cite{url:0-day-detection:darktrace}\cite{url:0-day-detection:wehowsky}, but are mostly limited to behavior analysis of the network using proprietary machine learning models. Some work, such as ZeroWall\cite{tang2020zerowall} exists for detection of vulnerabilities in web based applications.
\\
% add something here about NDR
Many security software vendors claim to have developed models and techniques to detect against Zero-day attacks\cite{url:0-day-detection:checkpoint}\cite{url:0-day-detection:capsule8}\cite{url:0-day-detection:logsign}, most of the work is proprietary and not available for study. Furthermore, based on publicly available information none of the vendors explain what telemetry is used other than \emph{``host and network based data''}.
\\
As with any other computer science field, machine learning has also been applied to exploitation of vulnerabilities. One example hereof is FastEmbed\cite{10.1371/journal.pone.0228439} which attempts to use machine learning to predict the number of exploits present in the wild, but is not related to detection of exploitation attempts.
\\
To our knowledge, not a lot of research can be found on using information gathered from patches to detect known vulnerabilities. Most research in this area revolves around using patch information to discover similar vulnerabilities\cite{xiao2020mvp}\cite{li2016vulpecker}

% \subimport{}{prerequisites.tex}

\end{document}